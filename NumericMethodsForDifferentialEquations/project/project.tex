\documentclass[11pt,a4paper]{article}
\usepackage[utf8]{inputenc}
\usepackage[bulgarian]{babel}
\usepackage{fullpage}
\usepackage{graphicx}
\usepackage{indentfirst}
\usepackage{hyperref}
\usepackage{xcolor}
\usepackage{amsmath}
\usepackage{mathtools}

\hypersetup{
    colorlinks,
    linkcolor={blue!55!black},
    citecolor={blue!50!black},
    urlcolor={blue!80!black}
}

\newcommand\Set[2]{\{\,#1\mid#2\,\}}

\begin{document}
\linespread{1.25}
\begin{titlepage}
   \begin{center}
       \vspace*{1cm}

\LARGE
       \textbf{Софийски университет „Св. Климент Охридски“\\
       Факултет по математика и инфоратика\\}

\addvspace{110pt}

	\Huge
       \textbf{Проект по Числени методи за диференциални уравнения}

 \vspace{0.5cm}
	\LARGE
        \textbf{Вариант 5}


       \vspace{1.5cm}
   \end{center}
    \vfill
\Large
       \textbf{Съставил:\\Ростислав Стоянов\\ф-н 45244}
 
    
 
 
       \vspace{0.8cm}
 
\end{titlepage}



\large
\tableofcontents
\pagebreak

\section{Условие на задачата}

Дадено ни е уравнението от втори ред:
$$ u''(x) -q(x)u(x) = f(x), x\in (a,b), $$
с гранични условия:
\begin{align*}
 \alpha_1u(a) + \beta_1\frac{\partial u}{\partial x}(a) = \mu_1, \\
\alpha_2u(b) + \beta_2\frac{\partial u}{\partial x}(b) = \mu_2.
\end{align*}
където $ a = 0,\  b = 1,\  q(x) = 1,\  f(x) = x^4 - 2x^3 -12x^2 +12x,\ \alpha_1 = 0,\ \beta_1 = 1,\\ \alpha_2 = 1,\ \beta_2 = 0,\ \mu_1 = 0 ,\ \mu_2 = -1.$

След заместване на параметрите в условието получаваме следната гранична задача за ОДУ от втори ред:
\begin{align}
 u''(x) - u(x) =x^4 - 2x^3 - 12x^2 + 12x&, x\in (0,1),\\
\frac{\partial u}{\partial x}(0) = 0,&\\ 
u(1) = -1&.
\end{align}
\\
Решението на задачата трябва да съдържа: 
\begin{itemize}
\item{ Диференчна схема с ЛГА О($h^2$),}
\item { Реализация на намерената диференчна схема използвайки метода\\ на прогонката,}
\item { Оценка на грешката чрез метода на Рунге с вложени мрежи.}
\end{itemize}

\pagebreak

\section{Построяване на диференчната схема}
Въвеждаме следната равномерна мрежа:
\begin{align*}
\overline{w_h} &\coloneqq  \Set{x_i  = ih }{h = \frac{1}{N}, i = 0,1\dots,N} ,
\end {align*}където h е големината на стъпката.

Първо намираме приближението на уравнението(1):
\begin{align*}
\frac{y_{i+1} - 2y_i + y_{i-1}}{h^2} - y_i &= {x_i}^4 - 2{x_i}^3 -12{x_i}^2 +12{x_i},
\end{align*}
където знаем, че апроксимацията на втората производна е с грешка О($h^2$).\\
Условието (3) се апроксимира точно : $y_N = -1$.\\
Остава да намерим приближение на условие (2). За целта ще използваме формула с разлика напред: 
\begin{align*}
\frac{y_1 - y_0}{h} &= 0.\\ 
\end{align*}
Пресмятаме ЛГА за тази апроксимация:
\begin{align*}
\psi_0 &= \frac{u_1 - u_0}{h} - 0 = \frac {1}{h}[u_0 + \frac{h}{1!}{u_0}' + \frac{h^2}{2!}{u_0}'' + O(h^3) -u_0] =\\  &={u_0}' + \frac{h}{2}{u_0}'' + O(h^2) = \frac{h}{2}{u_0}'' + O(h^2) 
\end{align*}
Допускаме, че уравнението се удовлетворява и при х = 0, т.е. изпълнено е $ u''(0) -u(0) = f(0) $. Тогава ${u_0}'' = u_0.$ Заместваме и получаваме
\begin{align*}
\psi_0 &= \frac{h}{2}{u_0} + O(h^2).
\end{align*}
Тогава приближението 
\begin{align*}
\frac{y_1 - y_0}{h} - \frac{h}{2}{y_0}  &= 0.\\ 
\end{align*}
ще има ЛГА O($h^2$).

Така търсената от нас диференчна схема има вида:

\begin{align*}
\frac{y_{i+1} - 2y_i + y_{i-1}}{h^2} - y_i = {x_i}^4 - 2{x_i}^3 -12{x_i}^2 +12{x_i},&\\
\frac{y_1 - y_0}{h} - \frac{h}{2}{y_0}  = 0,&\\
 y_N = -1&.
\end {align*}

\section{Метод на прогонката}
\end{document}